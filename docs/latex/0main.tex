 \documentclass[a4paper,twoside]{article}

%% Language and font encodings
\usepackage[spanish]{babel}
\usepackage[utf8]{inputenc}
\usepackage[T1]{fontenc}


%% Sets page size and margins
\usepackage[a4paper,top=3cm,bottom=2cm,left=2.5cm,right=2.5cm,marginparwidth=0.5cm]{geometry}

\usepackage{amsmath}			%Paquete matemático
\usepackage{graphicx}
\usepackage[colorinlistoftodos]{todonotes}

\usepackage{hyperref}		%Paquete empleado para colocar hipervinculos
\hypersetup{
	colorlinks = true,
	linkcolor = black,
}

\usepackage{eurosym}
\usepackage{pdfpages}			%Sirve para incluir PDF en el documento
\usepackage{anysize}			%Podremos colocar imagenes de cualquier tamaño
\usepackage{subfig}				%Nos permitira colocar varias imagenes en una figura
\usepackage{float}				%Podremos crear y colocar boxes donee queramos
\usepackage[export]{adjustbox}

%Colocamos cabeceras y pies de pagina
%(CONSULTA: http://edicionesoniricas.com/maquetar-latex-encabezados-pies-pagina/)
%(CONSULTA2: https://es.sharelatex.com/learn/Headers_and_footers)
%\bfseries es análogo a \textbf{}
% \leftmark-> Adds name and number of the current top-level structure (section for article) in uppercase letters.
%\rightmark-> Adds name and number of the current next to top-level structure (subsection for article) in uppercase letters.
\usepackage{fancyhdr}		%Paquetes necesarios
\pagestyle{fancy}			%Borra los parametros por defecto
\fancyhf{}
\fancyhead[RO,LE]{\bfseries\thepage}
\fancyhead[LO,RE]{\bfseries\rightmark}
%Nos aseguramos de que en las paginas plain, no haya ni cabeceras ni lineas
\fancypagestyle{plain}
{
	\fancyhead{} % elimina cabeceras en paginas "plain"
	\renewcommand{\headrulewidth}{0pt} % así como la raya
}

%Definimos las lineas divisoras de las cabeceras y pie de pagina
\renewcommand{\headrulewidth}{1pt}	%Define el grosor de la línea de head
\renewcommand{\footrulewidth}{0pt}		%Define el grosor de la linea foot (Si no queremos linea, 0pt)
\addtolength{\headheight}{0.5pt} % espacio para la raya

%Librerias para introducir código de Matlab
%\usepackage{bigfoot} % to allow verbatim in footnote
\usepackage[numbered,framed]{matlab-prettifier}

\lstset{
	style              = Matlab-editor,
	basicstyle         = \mlttfamily,
	escapechar         = ",
	mlshowsectionrules = true
}

% %%%%%%%%% INTRODUCIR CODIGO DE C %%%%%%%%%%%%%%%%%%%%%%
\usepackage{listings}
\usepackage{xcolor} % for setting colors

% set the default code style
%:Paquete para modificar los colores de diferentes elementos del codigo

\definecolor{mGreen}{rgb}{0,0.6,0}
\definecolor{mGray}{rgb}{0.5,0.5,0.5}
\definecolor{mPurple}{rgb}{0.58,0,0.82}
\definecolor{backgroundColour}{rgb}{0.95,0.95,0.92}

%Definimos el estilo del codigo de C
\lstdefinestyle{CStyle}{
	backgroundcolor=\color{backgroundColour},
	commentstyle=\color{mGreen},
	keywordstyle=\color{magenta},
	numberstyle=\tiny\color{mGray},
	stringstyle=\color{mPurple},
	basicstyle=\footnotesize,
	breakatwhitespace=false,
	breaklines=true,
	captionpos=b,
	keepspaces=true,
	numbers=left,
	numbersep=5pt,
	showspaces=false,
	showstringspaces=false,
	showtabs=false,
	tabsize=2,
	language=C,
}
% %%%%%%%%%%%%%%%%%%%%%%%%%%%%%%%%%%%%%%%%%%%%%%%%%%

% Pie de pagina
%\fancyfoot{} % limpia el pie
\fancyfoot[C]{- \thepage -} % número de página centrado

%Nos generará texto para pruebas de maquetado
\usepackage{lipsum}

% To include code
\usepackage{minted}
%\usemintedstyle{borland}
\usepackage{mdframed}

% To can use multirow
\usepackage{multirow}

% Se varia el limite de colimnas de latex
\setcounter{MaxMatrixCols}{11}
\usepackage{lscape}
%----------------------------------------------------------------------------------------------------------------------------------
\begin{document}
\begin{titlepage}
 \centering
 \Huge{\textbf{Péndulo Inverso controlado por voltantes de inercia}} \\
 \Huge{\textit{Trabajo Fin de Grado}}\\

 \vspace{1cm}
 \LARGE{Grado en Ingeniería Electrónica, Mecatrónica y Robótica}\\
 \rule{\textwidth}{0.1mm}
\Large{\tableofcontents}
 \vspace{2cm}
 \rule{\textwidth}{0.1mm}
 \Large{\textbf{Autor:} Haes-Ellis, Richard Mark}
\end{titlepage}
\newpage

% %%%%%%%%%%%   INTRODUCCION %%%%%%%%%%%%%%%%%%
\section{Introducción}
En este proyecto se desarrollara un HID(\textit{Human Interface Device}) para el control del puntero de un host, en este caso un ordenador, haciendo  uso del sensorpack \textit{BOOSTXL-SENSORS} y el microcontrolador \textit{Tiva TM4C1294}, ambos del fabricante \textit{Texas Instruments}.\\

\subsection{Descripcion del hardware empleado}
\subsection{Descripcion del software empleado}

\section{Funcionamiento del proyecto}
\begin{itemize}
\item Desarrollo de un HID con el microcontrolador.
\item Empleo del Boosterpack para tomar la definir la posicion del puntero en la pantalla.
\item Filtrado de las medidas tomadas a nivel de software.
\end{itemize}

\begin{itemize}
	\item \textbf{Host mode}: Permite conectar un teclado o un raton al microcontrolador.
	\item \textbf{Device mode}:Establece una comunicacion con el PC a traves del USB.
	\item \textbf{On-The-Go mode}: Permite multiplexar el USB entre hosts y dispositivos.
\end{itemize}

\section{Código de programación desarrollado}

\begin{itemize}
\item Instanciación de librerías
\item Declaración de variables
\item Funciones
\item Codigo principal
\end{itemize}

\subsection{Instanciacion de librerias}

\begin{listing}[h!]
\inputminted[linenos,breaklines,frame=lines,framesep=2mm]{c}{codes/librerias.c}
\caption{Instanciacion de librerías}
\end{listing}

\vspace{3cm}

\subsection{Funciones empleadas}
\begin{itemize}
  \item \textbf{SysTickIntHandler}:
    \begin{listing}[h!]
    \inputminted[linenos,breaklines,frame=lines,framesep=2mm]{c}{codes/fun_tick.c}
    \caption{Defines del código}
    \end{listing}

\vspace{4cm}

\end{mdframed}

\end{document}
